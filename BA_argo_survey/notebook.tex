
% Default to the notebook output style

    


% Inherit from the specified cell style.




    
\documentclass[11pt]{article}

    
    
    \usepackage[T1]{fontenc}
    % Nicer default font (+ math font) than Computer Modern for most use cases
    \usepackage{mathpazo}

    % Basic figure setup, for now with no caption control since it's done
    % automatically by Pandoc (which extracts ![](path) syntax from Markdown).
    \usepackage{graphicx}
    % We will generate all images so they have a width \maxwidth. This means
    % that they will get their normal width if they fit onto the page, but
    % are scaled down if they would overflow the margins.
    \makeatletter
    \def\maxwidth{\ifdim\Gin@nat@width>\linewidth\linewidth
    \else\Gin@nat@width\fi}
    \makeatother
    \let\Oldincludegraphics\includegraphics
    % Set max figure width to be 80% of text width, for now hardcoded.
    \renewcommand{\includegraphics}[1]{\Oldincludegraphics[width=.8\maxwidth]{#1}}
    % Ensure that by default, figures have no caption (until we provide a
    % proper Figure object with a Caption API and a way to capture that
    % in the conversion process - todo).
    \usepackage{caption}
    \DeclareCaptionLabelFormat{nolabel}{}
    \captionsetup{labelformat=nolabel}

    \usepackage{adjustbox} % Used to constrain images to a maximum size 
    \usepackage{xcolor} % Allow colors to be defined
    \usepackage{enumerate} % Needed for markdown enumerations to work
    \usepackage{geometry} % Used to adjust the document margins
    \usepackage{amsmath} % Equations
    \usepackage{amssymb} % Equations
    \usepackage{textcomp} % defines textquotesingle
    % Hack from http://tex.stackexchange.com/a/47451/13684:
    \AtBeginDocument{%
        \def\PYZsq{\textquotesingle}% Upright quotes in Pygmentized code
    }
    \usepackage{upquote} % Upright quotes for verbatim code
    \usepackage{eurosym} % defines \euro
    \usepackage[mathletters]{ucs} % Extended unicode (utf-8) support
    \usepackage[utf8x]{inputenc} % Allow utf-8 characters in the tex document
    \usepackage{fancyvrb} % verbatim replacement that allows latex
    \usepackage{grffile} % extends the file name processing of package graphics 
                         % to support a larger range 
    % The hyperref package gives us a pdf with properly built
    % internal navigation ('pdf bookmarks' for the table of contents,
    % internal cross-reference links, web links for URLs, etc.)
    \usepackage{hyperref}
    \usepackage{longtable} % longtable support required by pandoc >1.10
    \usepackage{booktabs}  % table support for pandoc > 1.12.2
    \usepackage[inline]{enumitem} % IRkernel/repr support (it uses the enumerate* environment)
    \usepackage[normalem]{ulem} % ulem is needed to support strikethroughs (\sout)
                                % normalem makes italics be italics, not underlines
    

    
    
    % Colors for the hyperref package
    \definecolor{urlcolor}{rgb}{0,.145,.698}
    \definecolor{linkcolor}{rgb}{.71,0.21,0.01}
    \definecolor{citecolor}{rgb}{.12,.54,.11}

    % ANSI colors
    \definecolor{ansi-black}{HTML}{3E424D}
    \definecolor{ansi-black-intense}{HTML}{282C36}
    \definecolor{ansi-red}{HTML}{E75C58}
    \definecolor{ansi-red-intense}{HTML}{B22B31}
    \definecolor{ansi-green}{HTML}{00A250}
    \definecolor{ansi-green-intense}{HTML}{007427}
    \definecolor{ansi-yellow}{HTML}{DDB62B}
    \definecolor{ansi-yellow-intense}{HTML}{B27D12}
    \definecolor{ansi-blue}{HTML}{208FFB}
    \definecolor{ansi-blue-intense}{HTML}{0065CA}
    \definecolor{ansi-magenta}{HTML}{D160C4}
    \definecolor{ansi-magenta-intense}{HTML}{A03196}
    \definecolor{ansi-cyan}{HTML}{60C6C8}
    \definecolor{ansi-cyan-intense}{HTML}{258F8F}
    \definecolor{ansi-white}{HTML}{C5C1B4}
    \definecolor{ansi-white-intense}{HTML}{A1A6B2}

    % commands and environments needed by pandoc snippets
    % extracted from the output of `pandoc -s`
    \providecommand{\tightlist}{%
      \setlength{\itemsep}{0pt}\setlength{\parskip}{0pt}}
    \DefineVerbatimEnvironment{Highlighting}{Verbatim}{commandchars=\\\{\}}
    % Add ',fontsize=\small' for more characters per line
    \newenvironment{Shaded}{}{}
    \newcommand{\KeywordTok}[1]{\textcolor[rgb]{0.00,0.44,0.13}{\textbf{{#1}}}}
    \newcommand{\DataTypeTok}[1]{\textcolor[rgb]{0.56,0.13,0.00}{{#1}}}
    \newcommand{\DecValTok}[1]{\textcolor[rgb]{0.25,0.63,0.44}{{#1}}}
    \newcommand{\BaseNTok}[1]{\textcolor[rgb]{0.25,0.63,0.44}{{#1}}}
    \newcommand{\FloatTok}[1]{\textcolor[rgb]{0.25,0.63,0.44}{{#1}}}
    \newcommand{\CharTok}[1]{\textcolor[rgb]{0.25,0.44,0.63}{{#1}}}
    \newcommand{\StringTok}[1]{\textcolor[rgb]{0.25,0.44,0.63}{{#1}}}
    \newcommand{\CommentTok}[1]{\textcolor[rgb]{0.38,0.63,0.69}{\textit{{#1}}}}
    \newcommand{\OtherTok}[1]{\textcolor[rgb]{0.00,0.44,0.13}{{#1}}}
    \newcommand{\AlertTok}[1]{\textcolor[rgb]{1.00,0.00,0.00}{\textbf{{#1}}}}
    \newcommand{\FunctionTok}[1]{\textcolor[rgb]{0.02,0.16,0.49}{{#1}}}
    \newcommand{\RegionMarkerTok}[1]{{#1}}
    \newcommand{\ErrorTok}[1]{\textcolor[rgb]{1.00,0.00,0.00}{\textbf{{#1}}}}
    \newcommand{\NormalTok}[1]{{#1}}
    
    % Additional commands for more recent versions of Pandoc
    \newcommand{\ConstantTok}[1]{\textcolor[rgb]{0.53,0.00,0.00}{{#1}}}
    \newcommand{\SpecialCharTok}[1]{\textcolor[rgb]{0.25,0.44,0.63}{{#1}}}
    \newcommand{\VerbatimStringTok}[1]{\textcolor[rgb]{0.25,0.44,0.63}{{#1}}}
    \newcommand{\SpecialStringTok}[1]{\textcolor[rgb]{0.73,0.40,0.53}{{#1}}}
    \newcommand{\ImportTok}[1]{{#1}}
    \newcommand{\DocumentationTok}[1]{\textcolor[rgb]{0.73,0.13,0.13}{\textit{{#1}}}}
    \newcommand{\AnnotationTok}[1]{\textcolor[rgb]{0.38,0.63,0.69}{\textbf{\textit{{#1}}}}}
    \newcommand{\CommentVarTok}[1]{\textcolor[rgb]{0.38,0.63,0.69}{\textbf{\textit{{#1}}}}}
    \newcommand{\VariableTok}[1]{\textcolor[rgb]{0.10,0.09,0.49}{{#1}}}
    \newcommand{\ControlFlowTok}[1]{\textcolor[rgb]{0.00,0.44,0.13}{\textbf{{#1}}}}
    \newcommand{\OperatorTok}[1]{\textcolor[rgb]{0.40,0.40,0.40}{{#1}}}
    \newcommand{\BuiltInTok}[1]{{#1}}
    \newcommand{\ExtensionTok}[1]{{#1}}
    \newcommand{\PreprocessorTok}[1]{\textcolor[rgb]{0.74,0.48,0.00}{{#1}}}
    \newcommand{\AttributeTok}[1]{\textcolor[rgb]{0.49,0.56,0.16}{{#1}}}
    \newcommand{\InformationTok}[1]{\textcolor[rgb]{0.38,0.63,0.69}{\textbf{\textit{{#1}}}}}
    \newcommand{\WarningTok}[1]{\textcolor[rgb]{0.38,0.63,0.69}{\textbf{\textit{{#1}}}}}
    
    
    % Define a nice break command that doesn't care if a line doesn't already
    % exist.
    \def\br{\hspace*{\fill} \\* }
    % Math Jax compatability definitions
    \def\gt{>}
    \def\lt{<}
    % Document parameters
    \title{survey}
    
    
    

    % Pygments definitions
    
\makeatletter
\def\PY@reset{\let\PY@it=\relax \let\PY@bf=\relax%
    \let\PY@ul=\relax \let\PY@tc=\relax%
    \let\PY@bc=\relax \let\PY@ff=\relax}
\def\PY@tok#1{\csname PY@tok@#1\endcsname}
\def\PY@toks#1+{\ifx\relax#1\empty\else%
    \PY@tok{#1}\expandafter\PY@toks\fi}
\def\PY@do#1{\PY@bc{\PY@tc{\PY@ul{%
    \PY@it{\PY@bf{\PY@ff{#1}}}}}}}
\def\PY#1#2{\PY@reset\PY@toks#1+\relax+\PY@do{#2}}

\expandafter\def\csname PY@tok@w\endcsname{\def\PY@tc##1{\textcolor[rgb]{0.73,0.73,0.73}{##1}}}
\expandafter\def\csname PY@tok@c\endcsname{\let\PY@it=\textit\def\PY@tc##1{\textcolor[rgb]{0.25,0.50,0.50}{##1}}}
\expandafter\def\csname PY@tok@cp\endcsname{\def\PY@tc##1{\textcolor[rgb]{0.74,0.48,0.00}{##1}}}
\expandafter\def\csname PY@tok@k\endcsname{\let\PY@bf=\textbf\def\PY@tc##1{\textcolor[rgb]{0.00,0.50,0.00}{##1}}}
\expandafter\def\csname PY@tok@kp\endcsname{\def\PY@tc##1{\textcolor[rgb]{0.00,0.50,0.00}{##1}}}
\expandafter\def\csname PY@tok@kt\endcsname{\def\PY@tc##1{\textcolor[rgb]{0.69,0.00,0.25}{##1}}}
\expandafter\def\csname PY@tok@o\endcsname{\def\PY@tc##1{\textcolor[rgb]{0.40,0.40,0.40}{##1}}}
\expandafter\def\csname PY@tok@ow\endcsname{\let\PY@bf=\textbf\def\PY@tc##1{\textcolor[rgb]{0.67,0.13,1.00}{##1}}}
\expandafter\def\csname PY@tok@nb\endcsname{\def\PY@tc##1{\textcolor[rgb]{0.00,0.50,0.00}{##1}}}
\expandafter\def\csname PY@tok@nf\endcsname{\def\PY@tc##1{\textcolor[rgb]{0.00,0.00,1.00}{##1}}}
\expandafter\def\csname PY@tok@nc\endcsname{\let\PY@bf=\textbf\def\PY@tc##1{\textcolor[rgb]{0.00,0.00,1.00}{##1}}}
\expandafter\def\csname PY@tok@nn\endcsname{\let\PY@bf=\textbf\def\PY@tc##1{\textcolor[rgb]{0.00,0.00,1.00}{##1}}}
\expandafter\def\csname PY@tok@ne\endcsname{\let\PY@bf=\textbf\def\PY@tc##1{\textcolor[rgb]{0.82,0.25,0.23}{##1}}}
\expandafter\def\csname PY@tok@nv\endcsname{\def\PY@tc##1{\textcolor[rgb]{0.10,0.09,0.49}{##1}}}
\expandafter\def\csname PY@tok@no\endcsname{\def\PY@tc##1{\textcolor[rgb]{0.53,0.00,0.00}{##1}}}
\expandafter\def\csname PY@tok@nl\endcsname{\def\PY@tc##1{\textcolor[rgb]{0.63,0.63,0.00}{##1}}}
\expandafter\def\csname PY@tok@ni\endcsname{\let\PY@bf=\textbf\def\PY@tc##1{\textcolor[rgb]{0.60,0.60,0.60}{##1}}}
\expandafter\def\csname PY@tok@na\endcsname{\def\PY@tc##1{\textcolor[rgb]{0.49,0.56,0.16}{##1}}}
\expandafter\def\csname PY@tok@nt\endcsname{\let\PY@bf=\textbf\def\PY@tc##1{\textcolor[rgb]{0.00,0.50,0.00}{##1}}}
\expandafter\def\csname PY@tok@nd\endcsname{\def\PY@tc##1{\textcolor[rgb]{0.67,0.13,1.00}{##1}}}
\expandafter\def\csname PY@tok@s\endcsname{\def\PY@tc##1{\textcolor[rgb]{0.73,0.13,0.13}{##1}}}
\expandafter\def\csname PY@tok@sd\endcsname{\let\PY@it=\textit\def\PY@tc##1{\textcolor[rgb]{0.73,0.13,0.13}{##1}}}
\expandafter\def\csname PY@tok@si\endcsname{\let\PY@bf=\textbf\def\PY@tc##1{\textcolor[rgb]{0.73,0.40,0.53}{##1}}}
\expandafter\def\csname PY@tok@se\endcsname{\let\PY@bf=\textbf\def\PY@tc##1{\textcolor[rgb]{0.73,0.40,0.13}{##1}}}
\expandafter\def\csname PY@tok@sr\endcsname{\def\PY@tc##1{\textcolor[rgb]{0.73,0.40,0.53}{##1}}}
\expandafter\def\csname PY@tok@ss\endcsname{\def\PY@tc##1{\textcolor[rgb]{0.10,0.09,0.49}{##1}}}
\expandafter\def\csname PY@tok@sx\endcsname{\def\PY@tc##1{\textcolor[rgb]{0.00,0.50,0.00}{##1}}}
\expandafter\def\csname PY@tok@m\endcsname{\def\PY@tc##1{\textcolor[rgb]{0.40,0.40,0.40}{##1}}}
\expandafter\def\csname PY@tok@gh\endcsname{\let\PY@bf=\textbf\def\PY@tc##1{\textcolor[rgb]{0.00,0.00,0.50}{##1}}}
\expandafter\def\csname PY@tok@gu\endcsname{\let\PY@bf=\textbf\def\PY@tc##1{\textcolor[rgb]{0.50,0.00,0.50}{##1}}}
\expandafter\def\csname PY@tok@gd\endcsname{\def\PY@tc##1{\textcolor[rgb]{0.63,0.00,0.00}{##1}}}
\expandafter\def\csname PY@tok@gi\endcsname{\def\PY@tc##1{\textcolor[rgb]{0.00,0.63,0.00}{##1}}}
\expandafter\def\csname PY@tok@gr\endcsname{\def\PY@tc##1{\textcolor[rgb]{1.00,0.00,0.00}{##1}}}
\expandafter\def\csname PY@tok@ge\endcsname{\let\PY@it=\textit}
\expandafter\def\csname PY@tok@gs\endcsname{\let\PY@bf=\textbf}
\expandafter\def\csname PY@tok@gp\endcsname{\let\PY@bf=\textbf\def\PY@tc##1{\textcolor[rgb]{0.00,0.00,0.50}{##1}}}
\expandafter\def\csname PY@tok@go\endcsname{\def\PY@tc##1{\textcolor[rgb]{0.53,0.53,0.53}{##1}}}
\expandafter\def\csname PY@tok@gt\endcsname{\def\PY@tc##1{\textcolor[rgb]{0.00,0.27,0.87}{##1}}}
\expandafter\def\csname PY@tok@err\endcsname{\def\PY@bc##1{\setlength{\fboxsep}{0pt}\fcolorbox[rgb]{1.00,0.00,0.00}{1,1,1}{\strut ##1}}}
\expandafter\def\csname PY@tok@kc\endcsname{\let\PY@bf=\textbf\def\PY@tc##1{\textcolor[rgb]{0.00,0.50,0.00}{##1}}}
\expandafter\def\csname PY@tok@kd\endcsname{\let\PY@bf=\textbf\def\PY@tc##1{\textcolor[rgb]{0.00,0.50,0.00}{##1}}}
\expandafter\def\csname PY@tok@kn\endcsname{\let\PY@bf=\textbf\def\PY@tc##1{\textcolor[rgb]{0.00,0.50,0.00}{##1}}}
\expandafter\def\csname PY@tok@kr\endcsname{\let\PY@bf=\textbf\def\PY@tc##1{\textcolor[rgb]{0.00,0.50,0.00}{##1}}}
\expandafter\def\csname PY@tok@bp\endcsname{\def\PY@tc##1{\textcolor[rgb]{0.00,0.50,0.00}{##1}}}
\expandafter\def\csname PY@tok@fm\endcsname{\def\PY@tc##1{\textcolor[rgb]{0.00,0.00,1.00}{##1}}}
\expandafter\def\csname PY@tok@vc\endcsname{\def\PY@tc##1{\textcolor[rgb]{0.10,0.09,0.49}{##1}}}
\expandafter\def\csname PY@tok@vg\endcsname{\def\PY@tc##1{\textcolor[rgb]{0.10,0.09,0.49}{##1}}}
\expandafter\def\csname PY@tok@vi\endcsname{\def\PY@tc##1{\textcolor[rgb]{0.10,0.09,0.49}{##1}}}
\expandafter\def\csname PY@tok@vm\endcsname{\def\PY@tc##1{\textcolor[rgb]{0.10,0.09,0.49}{##1}}}
\expandafter\def\csname PY@tok@sa\endcsname{\def\PY@tc##1{\textcolor[rgb]{0.73,0.13,0.13}{##1}}}
\expandafter\def\csname PY@tok@sb\endcsname{\def\PY@tc##1{\textcolor[rgb]{0.73,0.13,0.13}{##1}}}
\expandafter\def\csname PY@tok@sc\endcsname{\def\PY@tc##1{\textcolor[rgb]{0.73,0.13,0.13}{##1}}}
\expandafter\def\csname PY@tok@dl\endcsname{\def\PY@tc##1{\textcolor[rgb]{0.73,0.13,0.13}{##1}}}
\expandafter\def\csname PY@tok@s2\endcsname{\def\PY@tc##1{\textcolor[rgb]{0.73,0.13,0.13}{##1}}}
\expandafter\def\csname PY@tok@sh\endcsname{\def\PY@tc##1{\textcolor[rgb]{0.73,0.13,0.13}{##1}}}
\expandafter\def\csname PY@tok@s1\endcsname{\def\PY@tc##1{\textcolor[rgb]{0.73,0.13,0.13}{##1}}}
\expandafter\def\csname PY@tok@mb\endcsname{\def\PY@tc##1{\textcolor[rgb]{0.40,0.40,0.40}{##1}}}
\expandafter\def\csname PY@tok@mf\endcsname{\def\PY@tc##1{\textcolor[rgb]{0.40,0.40,0.40}{##1}}}
\expandafter\def\csname PY@tok@mh\endcsname{\def\PY@tc##1{\textcolor[rgb]{0.40,0.40,0.40}{##1}}}
\expandafter\def\csname PY@tok@mi\endcsname{\def\PY@tc##1{\textcolor[rgb]{0.40,0.40,0.40}{##1}}}
\expandafter\def\csname PY@tok@il\endcsname{\def\PY@tc##1{\textcolor[rgb]{0.40,0.40,0.40}{##1}}}
\expandafter\def\csname PY@tok@mo\endcsname{\def\PY@tc##1{\textcolor[rgb]{0.40,0.40,0.40}{##1}}}
\expandafter\def\csname PY@tok@ch\endcsname{\let\PY@it=\textit\def\PY@tc##1{\textcolor[rgb]{0.25,0.50,0.50}{##1}}}
\expandafter\def\csname PY@tok@cm\endcsname{\let\PY@it=\textit\def\PY@tc##1{\textcolor[rgb]{0.25,0.50,0.50}{##1}}}
\expandafter\def\csname PY@tok@cpf\endcsname{\let\PY@it=\textit\def\PY@tc##1{\textcolor[rgb]{0.25,0.50,0.50}{##1}}}
\expandafter\def\csname PY@tok@c1\endcsname{\let\PY@it=\textit\def\PY@tc##1{\textcolor[rgb]{0.25,0.50,0.50}{##1}}}
\expandafter\def\csname PY@tok@cs\endcsname{\let\PY@it=\textit\def\PY@tc##1{\textcolor[rgb]{0.25,0.50,0.50}{##1}}}

\def\PYZbs{\char`\\}
\def\PYZus{\char`\_}
\def\PYZob{\char`\{}
\def\PYZcb{\char`\}}
\def\PYZca{\char`\^}
\def\PYZam{\char`\&}
\def\PYZlt{\char`\<}
\def\PYZgt{\char`\>}
\def\PYZsh{\char`\#}
\def\PYZpc{\char`\%}
\def\PYZdl{\char`\$}
\def\PYZhy{\char`\-}
\def\PYZsq{\char`\'}
\def\PYZdq{\char`\"}
\def\PYZti{\char`\~}
% for compatibility with earlier versions
\def\PYZat{@}
\def\PYZlb{[}
\def\PYZrb{]}
\makeatother


    % Exact colors from NB
    \definecolor{incolor}{rgb}{0.0, 0.0, 0.5}
    \definecolor{outcolor}{rgb}{0.545, 0.0, 0.0}



    
    % Prevent overflowing lines due to hard-to-break entities
    \sloppy 
    % Setup hyperref package
    \hypersetup{
      breaklinks=true,  % so long urls are correctly broken across lines
      colorlinks=true,
      urlcolor=urlcolor,
      linkcolor=linkcolor,
      citecolor=citecolor,
      }
    % Slightly bigger margins than the latex defaults
    
    \geometry{verbose,tmargin=1in,bmargin=1in,lmargin=1in,rmargin=1in}
    
    

    \begin{document}
    
    
    \maketitle
    
    

    
    \begin{Verbatim}[commandchars=\\\{\}]
{\color{incolor}In [{\color{incolor}1}]:} \PY{k+kn}{import} \PY{n+nn}{pandas} \PY{k}{as} \PY{n+nn}{pd}
        \PY{k+kn}{from} \PY{n+nn}{matplotlib} \PY{k}{import} \PY{n}{pyplot} \PY{k}{as} \PY{n}{plt}
        \PY{o}{\PYZpc{}}\PY{k}{matplotlib} inline 
        \PY{k+kn}{import} \PY{n+nn}{numpy} \PY{k}{as} \PY{n+nn}{np}
        \PY{k+kn}{import} \PY{n+nn}{matplotlib}\PY{n+nn}{.}\PY{n+nn}{pyplot} \PY{k}{as} \PY{n+nn}{plt}
        \PY{n}{plt}\PY{o}{.}\PY{n}{style}\PY{o}{.}\PY{n}{use}\PY{p}{(}\PY{l+s+s1}{\PYZsq{}}\PY{l+s+s1}{seaborn\PYZhy{}white}\PY{l+s+s1}{\PYZsq{}}\PY{p}{)}
\end{Verbatim}


    \begin{Verbatim}[commandchars=\\\{\}]
{\color{incolor}In [{\color{incolor}2}]:} \PY{o}{!}ls
\end{Verbatim}


    \begin{Verbatim}[commandchars=\\\{\}]
 survey.ipynb  'Umfrage zu ArgoData.csv'

    \end{Verbatim}

    \begin{Verbatim}[commandchars=\\\{\}]
{\color{incolor}In [{\color{incolor}3}]:} \PY{n}{df} \PY{o}{=} \PY{n}{pd}\PY{o}{.}\PY{n}{read\PYZus{}csv}\PY{p}{(}\PY{l+s+s1}{\PYZsq{}}\PY{l+s+s1}{./Umfrage zu ArgoData.csv}\PY{l+s+s1}{\PYZsq{}}\PY{p}{)}
        \PY{k}{del} \PY{n}{df}\PY{p}{[}\PY{l+s+s1}{\PYZsq{}}\PY{l+s+s1}{Zeitstempel}\PY{l+s+s1}{\PYZsq{}}\PY{p}{]}
        \PY{n}{df}
\end{Verbatim}


\begin{Verbatim}[commandchars=\\\{\}]
{\color{outcolor}Out[{\color{outcolor}3}]:}     Ich kann mir sehr gut vorstellen, ArgoData regelmäßig zu nutzen.  \textbackslash{}
        0                                                   4                  
        1                                                   3                  
        2                                                   2                  
        3                                                   3                  
        4                                                   1                  
        5                                                   4                  
        6                                                   3                  
        7                                                   2                  
        8                                                   1                  
        9                                                   4                  
        10                                                  1                  
        11                                                  4                  
        
            Ich empfinde ArgoData als unnötig komplex.  \textbackslash{}
        0                                            1   
        1                                            2   
        2                                            2   
        3                                            1   
        4                                            3   
        5                                            1   
        6                                            1   
        7                                            3   
        8                                            3   
        9                                            1   
        10                                           1   
        11                                           1   
        
            Ich empfinde ArgoData als einfach zu nutzen.  \textbackslash{}
        0                                              4   
        1                                              3   
        2                                              5   
        3                                              5   
        4                                              3   
        5                                              5   
        6                                              5   
        7                                              3   
        8                                              4   
        9                                              5   
        10                                             5   
        11                                             4   
        
            Ich denke, dass ich Hilfestellung bei der Benutzung brauchen würde, um ArgoData zu nutzen.  \textbackslash{}
        0                                                   1                                            
        1                                                   4                                            
        2                                                   1                                            
        3                                                   3                                            
        4                                                   2                                            
        5                                                   4                                            
        6                                                   1                                            
        7                                                   3                                            
        8                                                   3                                            
        9                                                   3                                            
        10                                                  1                                            
        11                                                  2                                            
        
            Ich finde, dass die Funktionen (Karte, Steuerung, Anzeige von Daten) gut integriert sind.  \textbackslash{}
        0                                                   4                                           
        1                                                   4                                           
        2                                                   4                                           
        3                                                   5                                           
        4                                                   3                                           
        5                                                   5                                           
        6                                                   5                                           
        7                                                   4                                           
        8                                                   4                                           
        9                                                   5                                           
        10                                                  4                                           
        11                                                  5                                           
        
            Ich finde, dass es in ArgoData zu viele Inkonsistenzen gibt.  \textbackslash{}
        0                                                   3              
        1                                                   2              
        2                                                   3              
        3                                                   2              
        4                                                   2              
        5                                                   2              
        6                                                   1              
        7                                                   3              
        8                                                   3              
        9                                                   2              
        10                                                  1              
        11                                                  1              
        
            Ich kann mir vorstellen, dass die meisten Leute ArgoData schnell zu beherrschen lernen.  \textbackslash{}
        0                                                   4                                         
        1                                                   4                                         
        2                                                   5                                         
        3                                                   5                                         
        4                                                   4                                         
        5                                                   5                                         
        6                                                   5                                         
        7                                                   3                                         
        8                                                   4                                         
        9                                                   4                                         
        10                                                  5                                         
        11                                                  5                                         
        
            Ich empfinde die Bedienung als sehr umständlich.  \textbackslash{}
        0                                                  2   
        1                                                  2   
        2                                                  2   
        3                                                  2   
        4                                                  1   
        5                                                  2   
        6                                                  1   
        7                                                  2   
        8                                                  3   
        9                                                  1   
        10                                                 2   
        11                                                 1   
        
            Ich habe mich bei der Nutzung von ArgoData sehr sicher gefühlt.  \textbackslash{}
        0                                                   4                 
        1                                                   5                 
        2                                                   5                 
        3                                                   4                 
        4                                                   4                 
        5                                                   4                 
        6                                                   5                 
        7                                                   3                 
        8                                                   2                 
        9                                                   5                 
        10                                                  4                 
        11                                                  4                 
        
            Ich musste eine Menge Dinge lernen, bevor ich mit ArgoData arbeiten konnte.  
        0                                                   1                            
        1                                                   1                            
        2                                                   1                            
        3                                                   3                            
        4                                                   1                            
        5                                                   1                            
        6                                                   1                            
        7                                                   2                            
        8                                                   4                            
        9                                                   1                            
        10                                                  1                            
        11                                                  1                            
\end{Verbatim}
            
    \begin{Verbatim}[commandchars=\\\{\}]
{\color{incolor}In [{\color{incolor}4}]:} \PY{n}{fig}\PY{p}{,} \PY{n}{ax} \PY{o}{=} \PY{n}{plt}\PY{o}{.}\PY{n}{subplots}\PY{p}{(}\PY{p}{)}
        \PY{n}{fig}\PY{o}{.}\PY{n}{set\PYZus{}figwidth}\PY{p}{(}\PY{l+m+mi}{20}\PY{p}{)}
        \PY{n}{fig}\PY{o}{.}\PY{n}{set\PYZus{}figheight}\PY{p}{(}\PY{l+m+mf}{6.3}\PY{p}{)}
        
        
        \PY{n}{data} \PY{o}{=} \PY{n}{df}\PY{p}{[}\PY{l+s+s2}{\PYZdq{}}\PY{l+s+s2}{Ich kann mir sehr gut vorstellen, ArgoData regelmäßig zu nutzen.}\PY{l+s+s2}{\PYZdq{}}\PY{p}{]}
        
        \PY{n}{x} \PY{o}{=} \PY{n+nb}{list}\PY{p}{(}\PY{n+nb}{range}\PY{p}{(}\PY{n+nb}{len}\PY{p}{(}\PY{n}{data}\PY{p}{)}\PY{p}{)}\PY{p}{)}
        \PY{n}{y} \PY{o}{=} \PY{n}{np}\PY{o}{.}\PY{n}{array}\PY{p}{(}\PY{n}{data}\PY{p}{)}
        
        \PY{n}{\PYZus{}min} \PY{o}{=} \PY{n}{data}\PY{o}{.}\PY{n}{min}\PY{p}{(}\PY{p}{)}
        \PY{n}{\PYZus{}max} \PY{o}{=} \PY{n}{data}\PY{o}{.}\PY{n}{max}\PY{p}{(}\PY{p}{)}
        \PY{n}{\PYZus{}mean} \PY{o}{=} \PY{n}{data}\PY{o}{.}\PY{n}{mean}\PY{p}{(}\PY{p}{)}
        \PY{n}{\PYZus{}std} \PY{o}{=} \PY{n}{data}\PY{o}{.}\PY{n}{std}\PY{p}{(}\PY{p}{)}
        
        \PY{n}{ax}\PY{o}{.}\PY{n}{hist}\PY{p}{(}\PY{n}{y}\PY{p}{,} \PY{n}{bins}\PY{o}{=}\PY{p}{[}\PY{l+m+mi}{1}\PY{p}{,}\PY{l+m+mi}{2}\PY{p}{,}\PY{l+m+mi}{3}\PY{p}{,}\PY{l+m+mi}{4}\PY{p}{,}\PY{l+m+mi}{5}\PY{p}{]}\PY{p}{,} \PY{n}{histtype}\PY{o}{=}\PY{l+s+s1}{\PYZsq{}}\PY{l+s+s1}{bar}\PY{l+s+s1}{\PYZsq{}}\PY{p}{,} \PY{n}{align}\PY{o}{=}\PY{l+s+s1}{\PYZsq{}}\PY{l+s+s1}{left}\PY{l+s+s1}{\PYZsq{}}\PY{p}{,} \PY{n}{rwidth}\PY{o}{=}\PY{l+m+mf}{0.5}\PY{p}{)}
        \PY{c+c1}{\PYZsh{}ax.boxplot(data, showbox=True, vert=False)}
\end{Verbatim}


\begin{Verbatim}[commandchars=\\\{\}]
{\color{outcolor}Out[{\color{outcolor}4}]:} (array([ 3.,  2.,  3.,  4.]),
         array([1, 2, 3, 4, 5]),
         <a list of 4 Patch objects>)
\end{Verbatim}
            
    \begin{center}
    \adjustimage{max size={0.9\linewidth}{0.9\paperheight}}{output_3_1.png}
    \end{center}
    { \hspace*{\fill} \\}
    
    \begin{Verbatim}[commandchars=\\\{\}]
{\color{incolor}In [{\color{incolor}5}]:} \PY{n+nb}{print}\PY{p}{(}\PY{n}{plt}\PY{o}{.}\PY{n}{style}\PY{o}{.}\PY{n}{available}\PY{p}{)}
\end{Verbatim}


    \begin{Verbatim}[commandchars=\\\{\}]
['fivethirtyeight', 'seaborn-white', 'seaborn-whitegrid', 'seaborn-notebook', 'seaborn-dark', 'classic', 'bmh', 'seaborn-poster', 'seaborn-deep', 'seaborn-darkgrid', 'seaborn-pastel', 'grayscale', 'seaborn-muted', 'seaborn-colorblind', 'ggplot', 'seaborn', 'seaborn-dark-palette', 'dark\_background', 'seaborn-bright', '\_classic\_test', 'seaborn-paper', 'seaborn-ticks', 'seaborn-talk']

    \end{Verbatim}

    \begin{Verbatim}[commandchars=\\\{\}]
{\color{incolor}In [{\color{incolor}6}]:} plt.boxplot\PY{o}{??}
\end{Verbatim}


    \begin{Verbatim}[commandchars=\\\{\}]
{\color{incolor}In [{\color{incolor}17}]:} \PY{k+kn}{import} \PY{n+nn}{numpy} \PY{k}{as} \PY{n+nn}{np}
         \PY{k+kn}{import} \PY{n+nn}{seaborn} \PY{k}{as} \PY{n+nn}{sns}
         \PY{k+kn}{import} \PY{n+nn}{matplotlib}\PY{n+nn}{.}\PY{n+nn}{pyplot} \PY{k}{as} \PY{n+nn}{plt}
         \PY{k+kn}{import} \PY{n+nn}{matplotlib}\PY{n+nn}{.}\PY{n+nn}{patches} \PY{k}{as} \PY{n+nn}{patches}
         
         
         \PY{n}{plt}\PY{o}{.}\PY{n}{style}\PY{o}{.}\PY{n}{use}\PY{p}{(}\PY{l+s+s1}{\PYZsq{}}\PY{l+s+s1}{fivethirtyeight}\PY{l+s+s1}{\PYZsq{}}\PY{p}{)}
         
         \PY{k}{def} \PY{n+nf}{subplot}\PY{p}{(}\PY{n}{data}\PY{p}{)}\PY{p}{:}
             \PY{n}{bins}\PY{o}{=}\PY{p}{[}\PY{l+m+mi}{1}\PY{p}{,}\PY{l+m+mi}{2}\PY{p}{,}\PY{l+m+mi}{3}\PY{p}{,}\PY{l+m+mi}{4}\PY{p}{,}\PY{l+m+mi}{5}\PY{p}{]}
             \PY{n}{x} \PY{o}{=} \PY{p}{[}\PY{n}{data}\PY{o}{.}\PY{n}{value\PYZus{}counts}\PY{p}{(}\PY{p}{)}\PY{p}{[}\PY{n}{x}\PY{p}{]} \PY{k}{if} \PY{n}{x} \PY{o+ow}{in} \PY{n}{data}\PY{o}{.}\PY{n}{value\PYZus{}counts}\PY{p}{(}\PY{p}{)} \PY{k}{else} \PY{l+m+mi}{0} \PY{k}{for} \PY{n}{x} \PY{o+ow}{in} \PY{n}{bins}\PY{p}{]}
         
             \PY{n}{f}\PY{p}{,} \PY{p}{(}\PY{n}{ax\PYZus{}box}\PY{p}{,} \PY{n}{ax\PYZus{}hist}\PY{p}{)} \PY{o}{=} \PY{n}{plt}\PY{o}{.}\PY{n}{subplots}\PY{p}{(}\PY{l+m+mi}{2}\PY{p}{,} \PY{n}{sharex}\PY{o}{=}\PY{k+kc}{True}\PY{p}{,} \PY{n}{gridspec\PYZus{}kw}\PY{o}{=}\PY{p}{\PYZob{}}\PY{l+s+s2}{\PYZdq{}}\PY{l+s+s2}{height\PYZus{}ratios}\PY{l+s+s2}{\PYZdq{}}\PY{p}{:} \PY{p}{(}\PY{o}{.}\PY{l+m+mi}{1}\PY{p}{,} \PY{o}{.}\PY{l+m+mi}{85}\PY{p}{)}\PY{p}{\PYZcb{}}\PY{p}{)}
             \PY{n}{f}\PY{o}{.}\PY{n}{set\PYZus{}figwidth}\PY{p}{(}\PY{l+m+mi}{20}\PY{p}{)}
             \PY{n}{f}\PY{o}{.}\PY{n}{set\PYZus{}figheight}\PY{p}{(}\PY{l+m+mi}{4}\PY{p}{)}
         
             \PY{n}{whisker} \PY{o}{=} \PY{n}{sns}\PY{o}{.}\PY{n}{boxplot}\PY{p}{(}\PY{n}{data}\PY{o}{.}\PY{n}{values}\PY{p}{,} 
                                   \PY{n}{ax}\PY{o}{=}\PY{n}{ax\PYZus{}box}\PY{p}{,}\PY{n}{showmeans}\PY{o}{=}\PY{k+kc}{True}\PY{p}{,} 
                                   \PY{n}{meanline}\PY{o}{=}\PY{k+kc}{True}\PY{p}{,} \PY{n}{color}\PY{o}{=}\PY{l+s+s2}{\PYZdq{}}\PY{l+s+s2}{white}\PY{l+s+s2}{\PYZdq{}}\PY{p}{,} 
                                   \PY{n}{meanprops}\PY{o}{=}\PY{n+nb}{dict}\PY{p}{(}\PY{n}{color}\PY{o}{=}\PY{l+s+s1}{\PYZsq{}}\PY{l+s+s1}{darkgrey}\PY{l+s+s1}{\PYZsq{}}\PY{p}{,} \PY{n}{linestyle}\PY{o}{=}\PY{l+s+s1}{\PYZsq{}}\PY{l+s+s1}{\PYZhy{}}\PY{l+s+s1}{\PYZsq{}}\PY{p}{,} \PY{n}{linewidth}\PY{o}{=}\PY{l+m+mf}{2.5}\PY{p}{,}\PY{p}{)}\PY{p}{)}
             \PY{n}{ax\PYZus{}hist}\PY{o}{.}\PY{n}{bar}\PY{p}{(}\PY{n}{bins}\PY{p}{,} \PY{n}{x}\PY{p}{,} \PY{n}{color}\PY{o}{=}\PY{l+s+s2}{\PYZdq{}}\PY{l+s+s2}{darkgrey}\PY{l+s+s2}{\PYZdq{}}\PY{p}{)}
             \PY{n}{ax\PYZus{}box}\PY{o}{.}\PY{n}{set}\PY{p}{(}\PY{n}{yticks}\PY{o}{=}\PY{p}{[}\PY{p}{]}\PY{p}{)}
         
             \PY{n}{sns}\PY{o}{.}\PY{n}{despine}\PY{p}{(}\PY{n}{ax}\PY{o}{=}\PY{n}{ax\PYZus{}hist}\PY{p}{)}
             \PY{n}{sns}\PY{o}{.}\PY{n}{despine}\PY{p}{(}\PY{n}{ax}\PY{o}{=}\PY{n}{ax\PYZus{}box}\PY{p}{,} \PY{n}{left}\PY{o}{=}\PY{k+kc}{False}\PY{p}{)}
         
             \PY{n}{f}\PY{o}{.}\PY{n}{suptitle}\PY{p}{(}\PY{n}{data}\PY{o}{.}\PY{n}{name}\PY{p}{,} \PY{n}{fontsize}\PY{o}{=}\PY{l+m+mi}{20}\PY{p}{)}   
             
             \PY{n}{left}\PY{p}{,} \PY{n}{width} \PY{o}{=} \PY{o}{\PYZhy{}}\PY{o}{.}\PY{l+m+mi}{07}\PY{p}{,} \PY{l+m+mf}{1.05}
             \PY{n}{bottom}\PY{p}{,} \PY{n}{height} \PY{o}{=} \PY{o}{\PYZhy{}}\PY{l+m+mf}{0.09}\PY{p}{,} \PY{o}{.}\PY{l+m+mi}{5}
             \PY{n}{right} \PY{o}{=} \PY{n}{left} \PY{o}{+} \PY{n}{width}
             \PY{n}{top} \PY{o}{=} \PY{n}{bottom} \PY{o}{+} \PY{n}{height}
             
             \PY{n}{p} \PY{o}{=} \PY{n}{patches}\PY{o}{.}\PY{n}{Rectangle}\PY{p}{(}
             \PY{p}{(}\PY{n}{left}\PY{p}{,} \PY{n}{bottom}\PY{p}{)}\PY{p}{,} \PY{n}{width}\PY{p}{,} \PY{n}{height}\PY{p}{,}
             \PY{n}{fill}\PY{o}{=}\PY{k+kc}{False}\PY{p}{,} \PY{n}{transform}\PY{o}{=}\PY{n}{ax\PYZus{}hist}\PY{o}{.}\PY{n}{transAxes}\PY{p}{,} \PY{n}{clip\PYZus{}on}\PY{o}{=}\PY{k+kc}{False}
             \PY{p}{)}
             
             \PY{n}{ax\PYZus{}hist}\PY{o}{.}\PY{n}{text}\PY{p}{(}\PY{n}{right}\PY{p}{,} \PY{n}{height}\PY{p}{,} \PY{l+s+s1}{\PYZsq{}}\PY{l+s+s1}{sehr}\PY{l+s+s1}{\PYZsq{}}\PY{p}{,}
                 \PY{n}{horizontalalignment}\PY{o}{=}\PY{l+s+s1}{\PYZsq{}}\PY{l+s+s1}{center}\PY{l+s+s1}{\PYZsq{}}\PY{p}{,}
                 \PY{n}{verticalalignment}\PY{o}{=}\PY{l+s+s1}{\PYZsq{}}\PY{l+s+s1}{top}\PY{l+s+s1}{\PYZsq{}}\PY{p}{,}
                 \PY{n}{transform}\PY{o}{=}\PY{n}{ax}\PY{o}{.}\PY{n}{transAxes}\PY{p}{,}
                 \PY{n}{fontsize}\PY{o}{=}\PY{l+m+mi}{18}\PY{p}{)}
             
             \PY{n}{ax\PYZus{}hist}\PY{o}{.}\PY{n}{text}\PY{p}{(}\PY{n}{left}\PY{p}{,} \PY{n}{height}\PY{p}{,} \PY{l+s+s1}{\PYZsq{}}\PY{l+s+s1}{gar nicht}\PY{l+s+s1}{\PYZsq{}}\PY{p}{,}
                 \PY{n}{horizontalalignment}\PY{o}{=}\PY{l+s+s1}{\PYZsq{}}\PY{l+s+s1}{center}\PY{l+s+s1}{\PYZsq{}}\PY{p}{,}
                 \PY{n}{verticalalignment}\PY{o}{=}\PY{l+s+s1}{\PYZsq{}}\PY{l+s+s1}{top}\PY{l+s+s1}{\PYZsq{}}\PY{p}{,}
                 \PY{n}{transform}\PY{o}{=}\PY{n}{ax}\PY{o}{.}\PY{n}{transAxes}\PY{p}{,}
                 \PY{n}{fontsize}\PY{o}{=}\PY{l+m+mi}{18}\PY{p}{)}
         
         
         \PY{k}{for} \PY{n}{col} \PY{o+ow}{in} \PY{n}{df}\PY{p}{:}
             \PY{n}{subplot}\PY{p}{(}\PY{n}{df}\PY{p}{[}\PY{n}{col}\PY{p}{]}\PY{p}{)}
\end{Verbatim}


    \begin{center}
    \adjustimage{max size={0.9\linewidth}{0.9\paperheight}}{output_6_0.png}
    \end{center}
    { \hspace*{\fill} \\}
    
    \begin{center}
    \adjustimage{max size={0.9\linewidth}{0.9\paperheight}}{output_6_1.png}
    \end{center}
    { \hspace*{\fill} \\}
    
    \begin{center}
    \adjustimage{max size={0.9\linewidth}{0.9\paperheight}}{output_6_2.png}
    \end{center}
    { \hspace*{\fill} \\}
    
    \begin{center}
    \adjustimage{max size={0.9\linewidth}{0.9\paperheight}}{output_6_3.png}
    \end{center}
    { \hspace*{\fill} \\}
    
    \begin{center}
    \adjustimage{max size={0.9\linewidth}{0.9\paperheight}}{output_6_4.png}
    \end{center}
    { \hspace*{\fill} \\}
    
    \begin{center}
    \adjustimage{max size={0.9\linewidth}{0.9\paperheight}}{output_6_5.png}
    \end{center}
    { \hspace*{\fill} \\}
    
    \begin{center}
    \adjustimage{max size={0.9\linewidth}{0.9\paperheight}}{output_6_6.png}
    \end{center}
    { \hspace*{\fill} \\}
    
    \begin{center}
    \adjustimage{max size={0.9\linewidth}{0.9\paperheight}}{output_6_7.png}
    \end{center}
    { \hspace*{\fill} \\}
    
    \begin{center}
    \adjustimage{max size={0.9\linewidth}{0.9\paperheight}}{output_6_8.png}
    \end{center}
    { \hspace*{\fill} \\}
    
    \begin{center}
    \adjustimage{max size={0.9\linewidth}{0.9\paperheight}}{output_6_9.png}
    \end{center}
    { \hspace*{\fill} \\}
    
    \begin{Verbatim}[commandchars=\\\{\}]
{\color{incolor}In [{\color{incolor}8}]:} plt.bar\PY{o}{?}
\end{Verbatim}


    \begin{Verbatim}[commandchars=\\\{\}]
{\color{incolor}In [{\color{incolor}9}]:} \PY{n}{df}\PY{o}{.}\PY{n}{shape}\PY{p}{[}\PY{l+m+mi}{0}\PY{p}{]}
\end{Verbatim}


\begin{Verbatim}[commandchars=\\\{\}]
{\color{outcolor}Out[{\color{outcolor}9}]:} 12
\end{Verbatim}
            
    \begin{Verbatim}[commandchars=\\\{\}]
{\color{incolor}In [{\color{incolor}10}]:} \PY{n}{np}\PY{o}{.}\PY{n}{arange}\PY{p}{(}\PY{l+m+mi}{1}\PY{p}{)}
\end{Verbatim}


\begin{Verbatim}[commandchars=\\\{\}]
{\color{outcolor}Out[{\color{outcolor}10}]:} array([0])
\end{Verbatim}
            
    \begin{Verbatim}[commandchars=\\\{\}]
{\color{incolor}In [{\color{incolor}11}]:} \PY{n}{df}\PY{o}{.}\PY{n}{mean}\PY{p}{(}\PY{n}{axis}\PY{o}{=}\PY{l+m+mi}{0}\PY{p}{)}
\end{Verbatim}


\begin{Verbatim}[commandchars=\\\{\}]
{\color{outcolor}Out[{\color{outcolor}11}]:} Ich kann mir sehr gut vorstellen, ArgoData regelmäßig zu nutzen.                              2.666667
         Ich empfinde ArgoData als unnötig komplex.                                                    1.666667
         Ich empfinde ArgoData als einfach zu nutzen.                                                  4.250000
         Ich denke, dass ich Hilfestellung bei der Benutzung brauchen würde, um ArgoData zu nutzen.    2.333333
         Ich finde, dass die Funktionen (Karte, Steuerung, Anzeige von Daten) gut integriert sind.     4.333333
         Ich finde, dass es in ArgoData zu viele Inkonsistenzen gibt.                                  2.083333
         Ich kann mir vorstellen, dass die meisten Leute ArgoData schnell zu beherrschen lernen.       4.416667
         Ich empfinde die Bedienung als sehr umständlich.                                              1.750000
         Ich habe mich bei der Nutzung von ArgoData sehr sicher gefühlt.                               4.083333
         Ich musste eine Menge Dinge lernen, bevor ich mit ArgoData arbeiten konnte.                   1.500000
         dtype: float64
\end{Verbatim}
            
    \begin{Verbatim}[commandchars=\\\{\}]
{\color{incolor}In [{\color{incolor}12}]:} \PY{n}{df}\PY{o}{.}\PY{n}{std}\PY{p}{(}\PY{n}{axis}\PY{o}{=}\PY{l+m+mi}{0}\PY{p}{)}
\end{Verbatim}


\begin{Verbatim}[commandchars=\\\{\}]
{\color{outcolor}Out[{\color{outcolor}12}]:} Ich kann mir sehr gut vorstellen, ArgoData regelmäßig zu nutzen.                              1.230915
         Ich empfinde ArgoData als unnötig komplex.                                                    0.887625
         Ich empfinde ArgoData als einfach zu nutzen.                                                  0.866025
         Ich denke, dass ich Hilfestellung bei der Benutzung brauchen würde, um ArgoData zu nutzen.    1.154701
         Ich finde, dass die Funktionen (Karte, Steuerung, Anzeige von Daten) gut integriert sind.     0.651339
         Ich finde, dass es in ArgoData zu viele Inkonsistenzen gibt.                                  0.792961
         Ich kann mir vorstellen, dass die meisten Leute ArgoData schnell zu beherrschen lernen.       0.668558
         Ich empfinde die Bedienung als sehr umständlich.                                              0.621582
         Ich habe mich bei der Nutzung von ArgoData sehr sicher gefühlt.                               0.900337
         Ich musste eine Menge Dinge lernen, bevor ich mit ArgoData arbeiten konnte.                   1.000000
         dtype: float64
\end{Verbatim}
            
    \begin{Verbatim}[commandchars=\\\{\}]
{\color{incolor}In [{\color{incolor}13}]:} \PY{n}{df}\PY{o}{.}\PY{n}{max}\PY{p}{(}\PY{n}{axis}\PY{o}{=}\PY{l+m+mi}{0}\PY{p}{)}
\end{Verbatim}


\begin{Verbatim}[commandchars=\\\{\}]
{\color{outcolor}Out[{\color{outcolor}13}]:} Ich kann mir sehr gut vorstellen, ArgoData regelmäßig zu nutzen.                              4
         Ich empfinde ArgoData als unnötig komplex.                                                    3
         Ich empfinde ArgoData als einfach zu nutzen.                                                  5
         Ich denke, dass ich Hilfestellung bei der Benutzung brauchen würde, um ArgoData zu nutzen.    4
         Ich finde, dass die Funktionen (Karte, Steuerung, Anzeige von Daten) gut integriert sind.     5
         Ich finde, dass es in ArgoData zu viele Inkonsistenzen gibt.                                  3
         Ich kann mir vorstellen, dass die meisten Leute ArgoData schnell zu beherrschen lernen.       5
         Ich empfinde die Bedienung als sehr umständlich.                                              3
         Ich habe mich bei der Nutzung von ArgoData sehr sicher gefühlt.                               5
         Ich musste eine Menge Dinge lernen, bevor ich mit ArgoData arbeiten konnte.                   4
         dtype: int64
\end{Verbatim}
            
    \begin{Verbatim}[commandchars=\\\{\}]
{\color{incolor}In [{\color{incolor}14}]:} \PY{n}{df}\PY{o}{.}\PY{n}{min}\PY{p}{(}\PY{n}{axis}\PY{o}{=}\PY{l+m+mi}{0}\PY{p}{)}
\end{Verbatim}


\begin{Verbatim}[commandchars=\\\{\}]
{\color{outcolor}Out[{\color{outcolor}14}]:} Ich kann mir sehr gut vorstellen, ArgoData regelmäßig zu nutzen.                              1
         Ich empfinde ArgoData als unnötig komplex.                                                    1
         Ich empfinde ArgoData als einfach zu nutzen.                                                  3
         Ich denke, dass ich Hilfestellung bei der Benutzung brauchen würde, um ArgoData zu nutzen.    1
         Ich finde, dass die Funktionen (Karte, Steuerung, Anzeige von Daten) gut integriert sind.     3
         Ich finde, dass es in ArgoData zu viele Inkonsistenzen gibt.                                  1
         Ich kann mir vorstellen, dass die meisten Leute ArgoData schnell zu beherrschen lernen.       3
         Ich empfinde die Bedienung als sehr umständlich.                                              1
         Ich habe mich bei der Nutzung von ArgoData sehr sicher gefühlt.                               2
         Ich musste eine Menge Dinge lernen, bevor ich mit ArgoData arbeiten konnte.                   1
         dtype: int64
\end{Verbatim}
            

    % Add a bibliography block to the postdoc
    
    
    
    \end{document}
