\section{Thematik}


Seit dem Jahr 2000 untersucht das Argo-Projekt die Auswirkungen des Klimawandels auf die Meere des Planeten Erde.
Dabei messen inzwischen über 3900 Treibbojen den Salzgehalt und die Temperatur der oberen 2000 Meter tiefen Wassersäule der eisfreien Meere. In einem  Intervall von ca. 10 Tagen tauchen diese sogenannten Floats in die Tiefen ab und messen über einen Zeitraum von ungefähr sechs Stunden Leitfähigkeit, Temperatur und Druck des umliegenden Wassers. Nach dem Tauchgang werden die  ermittelten Daten inklusive der GPX-Koordinaten an das Satellitensystem Iason gesendet. Durch die Satelliten werden die Messdaten an Bodenstationen übertragen. Die gesammelten Daten  werden über mehrere Datenzentren vorgehalten und stehen ausdrücklich der freien Benutzung offen. \footnote{ vgl. Quelle: \cite{ArgoDataBeginnersGuide}}

Der von den Menschen verursachte Klimawandel mit all seinen abzusehenden Auswirkungen 
stößt in den letzten Jahren bei einigen Menschen verstärkt auf Skepsis. 
Dies könnte nicht zuletzt an der abstrakten Darstellung von Klimadaten liegen. Ein spielerischer und bildhafter Zugang zu den Messdaten könnte hier einen Beitrag leisten, 
diese Thematik bei einigen Menschen wieder in den Fokus zu rücken.
Die Daten der Messroboter des Argo-Projekts scheinen hierfür geeignet, da sie zum einen frei zugänglich sind, und zum anderen direkt die Auswirkungen des Treibhauseffektes auf die Weltmeeres zeigen. 

Durch die enthaltenen Geokoordinaten eignen sich die Daten für eine aggregierte Anzeige über einer durch Openstreetmap gerenderten Karte. Denkbar wären hierbei zwei Szenarien: 1) Es werden die Floats ausgewählt, die sich in der jeweiligen Zoomstufe des Kartenausschnittes des Browsers zu sehen sind. Der durchschnittliche Verlauf der Messdaten wird daraufhin ermittelt und als Diagramm neben der Karte gezeichnet.
2) Eine Messboje wird angeklickt, Daten der jeweiligen Boje werden als Diagramm neben der Karte angezeigt. 

Bereits heute gibt es eine Reihe von Betrachtern für die Argo-Daten. \footnote{ vgl. Quelle: \cite{ArgoDataViewers}}  
%Diese sind aber aufgrund ihrer Funktion als Data-Mining-Programme sehr mächtige Filter- und Agregations-Hilfen für die Arbeit von Wissenschaftlern \footnote{Beispiel: \url{http://www.ifremer.fr/lpo/naarc/}} und sind für einen schnellen Überblick nicht geeignet.
Konzipiert wurden diese Programme als Filter und Aggregationshilfen für die Arbeit von Wissenschaftlern.\footnote{Beispiel: \url{http://www.ifremer.fr/lpo/naarc/}} Für einen schnellen Überblick, ohne das benötigte Fachwissen, sind diese aber ungeeignet.

Ziel der Bachelorarbeit ist es, einen prototypischen geobasierten Betrachtungsdienst für das Argo-Projekt zu entwerfen.
