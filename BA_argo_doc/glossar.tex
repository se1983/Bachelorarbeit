
\newacronym{JCOMM}{JCOMM}{Joint Technical Commission for Oceanography and Marine Meteorology}
\newacronym{GDAC}{GDAC}{Global Data Access Committee}
\newacronym{GOOS}{GOOS}{Global Ocean Observing System}
\newacronym{DOI}{DOI}{digital object identifier}
\newacronym{CDF}{CDF}{Common Data Format}
\newacronym{netCDF}{netCDF}{Network Common Data Format}
\newacronym{DBMS}{DBMS}{Datenbank-Management-System}
\newacronym{ORM}{ORM}{Objektrelationaler Mapper}
\newacronym{KISS}{KISS}{Keep It Simple, Stupid}


\newglossaryentry{Controller}{%
    name={Controller},
    description={Element zur Programmsteuerung der Webanwendung}
}

\newglossaryentry{HTTP}{%
    name={HTTP},
    description={Hypertext Transfer Protocol. Weit verbreitetes Protokoll um beispielsweise das WWW in den Webbrowser zu laden}
}

\newglossaryentry{FTP}{%
    name={FTP},
    description={File Transfer Protocol. Protokoll zur Übermittlung von Daten}
}

\newglossaryentry{api}{%
    name={API},
    description={Application-Programming-Interface. Programmschnittstelle, die Daten in maschinenlesbarer Form darstellt}
}

\newglossaryentry{HTML}{%
    name={HTML},
    description={Hypertext Markup Language. Beschreibungssprache die häufig für die Inhaltsbeschreibung von Webseiten verwendet wird}
}

\newglossaryentry{TESAC}{%
  name={TESAC},
  description={Wissenschaftliches Datenformat zur Übertragung von Sensordaten}
}

\newglossaryentry{BUFR}{%
  name={BUFR},
  description={Binäres Datenformat zur Übetragung meterologischer Daten}
}

\newglossaryentry{Framework}{%
    name={Framework},
    description={Erweiterbare Softwarebibliothek zur Erstellung von Systemen. Funktioniert nach dem Hollywood-Prinzip (don't call us, we call you)}
}

\newglossaryentry{Julian Day}{%
    name={Julian Day},
    description={(Julianisches Datum) In der Wissenschaft gebräuchliches Datumsformat über die Tageszählung ab einem fixen Startdatum (meist 1. Januar 4713 v. Chr.)}
}

\newglossaryentry{nan}{%
    name={NaN},
    description={Not a Number. Ist ein numerischer Wert, der undefiniert ist, oder keinem Wert entspricht}
}

\newglossaryentry{WSGI}{%
    name={WSGI},
    description={Web Server Gateway Interface. Schnittstellenspezifikation zur Protikollstandardisierung der Kommonikation von Python-Programmen und Webservern}
}
