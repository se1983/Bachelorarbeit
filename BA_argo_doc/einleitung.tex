
\section{Einleitung}

    
    Die im Jahre 1965 von Gordon Moore vorhergesagte Gesätzmäßigkeit, das sich die Komplexität, und damit die Speicherdichte integrierter Schaltkreise regelmäßig verdoppeln,  hat sich bis zum heutigen Tag bewahrheitet. Damit sieht sich die Menschheit heute in einer einmaligen Lage. Wir besitzen über eine so noch nie dagewesene Ansammlung an Informationen. Durch die wachsende Komplexität der dafür benötigten Technik werden die Medien zur Speicherung auch immer flüchtiger und schwieriger in der Handhabung. Werden kommende Zivilisationen in der Lage sein, diesen Pool an Informationen für sich zu nutzen oder sollten wir vielmehr annehmen, dass wir eine einmalige Chance haben, die wir nicht vergeuden sollten? \\
    
    Was hält uns davon ab, Bildung und Wohlstand aus diesem Pool zu generieren? Viele der Informationen stehen frei zur Verfügung und können genutzt werden und doch werden gerade in dieser Zeit die Prinzipien der wissenschaftlichen Evidenz von einem gefühlt immer größeren Anteil unseres Kulturkreises abgelehnt. 
    In einer Zeit in der es einfacher den je ist, Fakten zu überprüfen, werden wissenschaftlich bewiesenen Aussagen wie dem Klimawandel einfach nicht geglaubt.
    Es scheint als wären viele Menschen der Flut an Informationen überdrüssig, als wende sich ein großer Teil überfordert davon ab.
    Die Frage ist, was kann dazu beitragen dieses Potential mehr zu nutzen? 
    Über welche Mittel verfügt die Informatik, Daten in einen Kontext einzubetten, über die Menschen an  die wissenschaftliche Arbeit herangeführt werden können? Welche Daten sind geeignet, um die Brisanz und das Potential unserer Zeit einem breiteren Publikum zuzuführen.
    
    Hier bietet das Argoprogramm eine Möglichkeit die hierfür benötigten Daten bereitzustellen. Unter dem Dach dieses Programms, werden seit anfang dieses Jahrtausends die Weltmeere nach den Parametern Temperatur, Salzgehalt und Leitfähigkeit untersucht. Diese Daten stehen unter einer freien Lizenz zur Verfügung und können in eigene Projekte eingebunden werden.  Diese dienen Wissenschaftlern um die Auswirkungen des globalen Klimawandels zu untersuchen.
   
    Ziel dieser Arbeit ist eine einfache Darstellung aus diese Messwerten zu erarbeiten. Über ein exploratives Werkzeug sollen die  die wissenschaftlichen Daten intuitiv erfahrbar sein. Dies geschieht mit der Hoffnung, hier eine Identifikation mit den Messwerten und des wissenschaftlichen Prozesses zu erreichen. 
   
   
    Diese Anwendung wäre zum Beispiel für Schulklassen geeignet. Die Schüler können durch eine Kontextvorgabe des Lehrenden  die Auswirkungen des Klimawandels auf die Weltmeere für sich explorieren. Zusätzlich könnte beim ein oder anderen Interesse an der wissenschaftlichen Arbeit geweckt werden.
    Das Angebot richtet sich aber nicht explizit an heranwachsende. Auch erwachsene Personen können sich hier bei Interesse weiterbilden. Die Applikation kann hier außerhalb der wissenschaftlichen Arbeit die Relevanz der Forschung und die in diesem Programm erhobenen Messwerte  erfahrbar machen. 
    
    Am Anfang dieser Arbeit wird das Argo Programm vorgestellt. es wird der Prozess der Datenerhebung und -veröffentlichung eingegangen. 
    Im Anschluss werden die Anforderungen für die zu entwickelte Software ausgearbeitet und damit die geeignetsten Werkzeuge ermittelt.
    Daraufhin wird das System entworfen. Es wird eine Architekturbeschreibung durchgeführt und die wichtigsten Geschäftprozesse beschrieben. Hier werden Alternativen aufgezeigt und versucht, die geeignetste zu ermitteln.
    Die Implementierung wird im darauf folgenden Kapitel beschrieben. Hier werden bestimmte Prozesse iterativ verfeinert und verbessert um die am besten geeignetste Lösung zu finden. 
    Um die Qualität der Software beschreiben zu können, folgt im darauf folgenden Kapitel eine Beschreibung der Verwendeten Test-Verfahren. Es werden die durchgeführten Unit-tests sowie eine Umfrage zur Bestimmung der Usability beschrieben.    
    Abschließend wird die fertige Software demonstriert. Hierbei wird versucht einen möglichst kritischen Blick auf das Projekt zu werfen und Alternativen und Verbesserungen vorzustellen.
    
    
    
\subsection{Existierende Lösungen}
  
\subsubsection{Das JCOMMops} 
   
   Das Joint Technical Commission for Oceanography and Marine Meteorology (JCOMM) bietet mit 
    \url{jcommops.org} eine Grundlage für die wissenschaftliche Arbeit unter anderem mit den von Argo gesammelten Daten. Neben der Darstellung von Karten erfährt der Nutzer hier, von Sensordaten über den Bautyp der jeweiligen Boje alles was die Bojen zu erzählen haben. Daneben werden über diese Plattform auch redaktionelle Reports veröffentlicht, um der Leserschaft ein Bild der aktuellen Lage unserer Weltmeere zu vermitteln. Über einen Twitter Account werden Änderungen an der Plattform und neu veröffentlichte Reports veröffentlicht. 
    
    Zwar bietet die Plattform durch ihre kartenbasierte explorative Darstellung ein ähnliches Angebot, wie es in ArgoData gemacht wird. Die schiere Fülle der Parameter erfordert vom Benutzer aber die Bereitschaft sich vertieft in die Matherie einzuarbeiten. Damit richtet sich das Angebot des Global Ocean Observing Systems an Wissenschaftler und Journalisten. Diesen dient sie als eine hervorragende Datengrundlage für deren Veröffentlichungen. Ein Benutzer aus der hier genannten Zielgruppendefinition wird von einem derartigen Angebot wohl eher abgeschreckt sein, bevor er dessen Vorteile für sich erarbeiten konnte. 
    
\subsubsection{Data Selection and Visualization Tool des Coriolis GDAC}
    
    Das Data selection and visualization tool der Coriolis GDAC \footnote { siehe \cite{ArgoDataSelection}} ermöglicht den Download und von nach verschiedenen Filterkriterien ausgewählten Datensätzen. Ein Betrachter erlaubt zusätzlich werte einer spezifischen Treibboje anzusehen und ihren bisher zurückgelegten Weg nachzuvollziehen.
    
    Auch wenn die Darstellung hier mit weniger Parameter auskommt, so ist die primäre Aufgabenstellung dieser Plattform wohl das extrahieren der Daten um diese in einem wissenschaftlichen Paper verwenden zu können. Neben der akademischen Verwendung müsste ein Benutzer aber auch hier viel Neugierde und Zeit für die Einarbeitung in das Thema mitbringen um aus den angebotenen Daten einen Mehrwert für sich zu generieren. 
    
    
\subsection{Alleinstellungsmerkmal \& Abgrenzung}
    
    Das Ziel dieser Arbeit ist die Darstellung von wissenschaftlichen Daten aus dem Argo-Programm. Im Gegensatz zu den bereits vorhandenen Lösungen sollen die Daten aber nicht zur wissenschaftlichen Verwendung aufbereitet werden. Vielmehr sollen die hochkomplexen Daten auf ein einfach erfahrbares Maß herunter gebrochen werden. Zum Zeitpunkt dieser Arbeit existiert noch kein exploratives Tool für Daten des Argo Programms mit diesem Ansatz.
    

