\documentclass[a4paper,12pt, ngerman,bibliography=totocnumbered ]{scrartcl}
\usepackage[a4paper]{geometry}


\geometry{a4paper,
        tmargin=3cm,
        bmargin=3cm,
        lmargin=3.6cm,
        rmargin=2.2cm,
        headheight=3em,
        headsep=2em,
        footskip=1cm}


\usepackage[onehalfspacing]{setspace}
\usepackage{epigraph}

\renewcommand{\familydefault}{\sfdefault}


\usepackage{amssymb}% http://ctan.org/pkg/amssymb
\usepackage{pifont}% http://ctan.org/pkg/pifont
\newcommand{\cmark}{\ding{51}}%
\newcommand{\xmark}{\ding{55}}%
\newcommand{\pmark}{\ding{48}}

\usepackage[utf8x]{inputenc}
\usepackage{graphicx}
\usepackage{float}

\usepackage{pdfpages}
\usepackage{wrapfig}

\PassOptionsToPackage{hyphens}{url}\usepackage[colorlinks,
pdfpagelabels,
urlcolor = black,
pdfstartview = FitH,
bookmarksopen = true,
bookmarksnumbered = true,
linkcolor = black,
plainpages = false,
hypertexnames = false,
citecolor = black,
breaklinks=true] {hyperref}




\usepackage[automark,headsepline]{scrlayer-scrpage}

\clearpairofpagestyles
\cfoot[\pagemark]{\pagemark}
\lehead{\headmark}
\rohead{\emph{\headmark}}
\setlength\parindent{0pt}

\definecolor{codeblack}{RGB}{83,83,83}
\definecolor{xdeepblack}{RGB}{64,64,64}

\renewcommand*{\headfont}{\color{xdeepblack}\bfseries}
\renewcommand*{\pnumfont}{\normalfont\bfseries}
\addtokomafont{paragraph}{\normalfont\color{xdeepblack}\bfseries}


\addtokomafont{descriptionlabel}{\color{xdeepblack}\normalfont\bfseries}

\definecolor{codegray}{rgb}{0.5,0.5,0.5}
\setheadsepline{1pt}[\color{codegray}]

%\urlstyle{tt}

\renewcommand{\UrlBreaks}{\do\/\do\a\do\b\do\c\do\d\do\e\do\f\do\g\do\h\do\i\do\j\do\k\do\l\do\m\do\n\do\o\do\p\do\q\do\r\do\s\do\t\do\u\do\v\do\w\do\x\do\y\do\z\do\A\do\B\do\C\do\D\do\E\do\F\do\G\do\H\do\I\do\J\do\K\do\L\do\M\do\N\do\O\do\P\do\Q\do\R\do\S\do\T\do\U\do\V\do\W\do\X\do\Y\do\Z}


\usepackage[ngerman]{babel}

%\addto\captionsngerman{\renewcommand{\refname}{Bisher gesichtete Literatur}}

\usepackage{setspace}
\usepackage[flushmargin]{footmisc}
%\usepackage{tikz}
\usepackage{wrapfig}
\usepackage{nameref}
%\usepackage{lmodern}

\usepackage[T1]{fontenc}
\usepackage{inconsolata}

\renewcommand\textbullet{\ensuremath{-}}

\usepackage[nottoc]{tocbibind}
\usepackage[verbose]{placeins}

% -------------------------------------------------------
%                   LISTINGS
%
%
% Default fixed font does not support bold face
\DeclareFixedFont{\ttb}{T1}{txtt}{b}{n}{10} % for bold
\DeclareFixedFont{\ttm}{T1}{txtt}{m}{n}{10}  % for normal
%
% Custom colors
\usepackage{color}
\definecolor{deepblue}{rgb}{0,0,0.5}
\definecolor{deepred}{rgb}{0.6,0,0}
\definecolor{deepgreen}{rgb}{0,0.5,0}
\definecolor{codegreen}{rgb}{0,0.6,0}
\definecolor{codebrown}{rgb}{0.3,0.3,0}
\definecolor{codegray}{rgb}{0.5,0.5,0.5}
\definecolor{codepurple}{rgb}{0.58,0,0.82}
\definecolor{backcolour}{RGB}{244,244,244}
\definecolor{codedarkgey}{RGB}{128,128,128}
\definecolor{codewhite}{RGB}{252,252,252}
\definecolor{codeblack}{RGB}{83,83,83}
%
\usepackage{listings}
%
% -------------------------------
% Python style for highlighting
%
\newcommand\pythonstyle{\lstset{
    language=Python,
    backgroundcolor = \color{codewhite},
    keepspaces=true,
    commentstyle=\color{codegray},
    stringstyle=\color{codedarkgey},
    rulesepcolor=\color{gray},
    rulecolor=\color{lightgray},
    otherkeywords={self, yield},
    postbreak=\mbox{\textcolor{lightgray}{$\hookrightarrow$}\space},
    keywordstyle=\ttb\color{codeblack},
    emphstyle=\ttb\color{codedarkgey},
    numberstyle=\small\color{lightgray},
    emph={ArgoFloat,__init__,__iter__, __next__,__enter__,__exit__, __repr__, Measurement, Location, Profile, Record, ExtractorFactory, DataBaseWriter, DatasetContextManager, ArgoFloatProfile, Float },
    frame=single,
    showstringspaces=false,
    breaklines=true,
    lineskip={-1.5pt},
    numbers=left,
    numbersep=5pt,
    stepnumber=1,
    captionpos=b,
    basicstyle=\ttfamily\footnotesize,
    aboveskip=20pt,
    belowskip=20pt,
}}


% Python environment
\lstnewenvironment{python}[1][]
{
\pythonstyle
\lstset{#1}
}
{}

% Python for external files
\newcommand\pythonexternal[2][]{{
\pythonstyle
\lstinputlisting[#1]{#2}}}

% Python for inline
\newcommand\pythoninline[1]{{\pythonstyle\lstinline!#1!}}


\lstdefinelanguage{JavaScript}{
  keywords={typeof, new, true, false, catch, function, return, null, catch, switch, var, if, in, while, do, else, case, break},
  keywordstyle=\color{blue}\bfseries,
  ndkeywords={class, export, boolean, throw, implements, import, this},
  ndkeywordstyle=\color{darkgray}\bfseries,
  identifierstyle=\color{black},
  sensitive=false,
  comment=[l]{//},
  morecomment=[s]{/*}{*/},
  commentstyle=\color{purple}\ttfamily,
  stringstyle=\color{red}\ttfamily,
  morestring=[b]',
  morestring=[b]"
}

\newcommand\jsstyle{\lstset{
    language=JavaScript,
    backgroundcolor = \color{codewhite},
    keepspaces=true,
    commentstyle=\color{codegray},
    stringstyle=\color{codedarkgey},
    rulesepcolor=\color{gray},
    rulecolor=\color{lightgray},
    otherkeywords={self, yield},
    postbreak=\mbox{\textcolor{lightgray}{$\hookrightarrow$}\space},
    keywordstyle=\ttb\color{codeblack},
    emphstyle=\ttb\color{codedarkgey},
    numberstyle=\small\color{lightgray},
    keepspaces=true,
    %commentstyle=\color{codebrown},
    %keywordstyle=\color{magenta},
    %stringstyle=\color{codepurple},
    otherkeywords={},             % Add keywords here
    %keywordstyle=\ttb\color{deepblue},
    emph={ArgoFloat,__init__,__iter__, __next__,__enter__,__exit__, __repr__, Measurement, Location, Profile, Record, ExtractorFactory, DataBaseWriter, DatasetContextManager, ArgoFloatProfile, Float },          % Custom highlighting
    %emphstyle=\ttb\color{deepred},    % Custom highlighting style
    frame=single,                         % Any extra options here
    showstringspaces=false,            %
    breaklines=true,
    lineskip={-1.5pt},
    numbers=left,
    numbersep=5pt,
    stepnumber=1,
    captionpos=b,
    basicstyle=\ttfamily\footnotesize,
    aboveskip=20pt,
    belowskip=20pt,
}}

\lstnewenvironment{javascript}[1][]
{
\jsstyle
\lstset{#1}
}
{}
\newcommand\jsinline[1]{{\jsstyle\lstinline!#1!}}

% -------------------------------------------------------------------------------------

\renewcaptionname{ngerman}{\contentsname}{Inhalt}
\renewcaptionname{ngerman}{\listfigurename}{Abbildungsverzeichnis}
\renewcaptionname{ngerman}{\listtablename}{Tabellenverzeichnis}
\renewcommand\lstlistlistingname{Quellcodeverzeichnis}
\renewcaptionname{ngerman}{\refname}{Quellenverzeichnis}
\renewcaptionname{ngerman}{\figurename}{Abbildung}
\renewcaptionname{ngerman}{\tablename}{Tabelle}
\setkomafont{caption}{\footnotesize\itshape}
\setkomafont{captionlabel}{\usekomafont{caption}}


\usepackage{booktabs}



\usepackage{glossaries}
\makenoidxglossaries

\newacronym{JCOMM}{JCOMM}{"`Joint Technical Commission for Oceanography and Marine Meteorology"'}

\newacronym{GDAC}{GDAC}{"`Data Selection and Visualization Tool"'}

\newacronym{GOOS}{GOOS}{Global Ocean Observing
System}

\newacronym{DOI}{DOI}{digital object identifier}

\newacronym{CDF}{CDF}{Common Data Format}
\newacronym{netCDF}{netCDF}{Network Common Data Format}

\newacronym{DBMS}{DBMS}{Datenbank-Management-System}

\newacronym{ORM}{ORM}{Objektrelationaler Mapper}




\newglossaryentry{Controller}{%
    name={Controller},
    description={Element zur Programmsteuerung der Webanwendung}
}

\newglossaryentry{HTTP}{%
    name={HTTP},
    description={Hypertext Transfer Protocol. Weit verbreitetes Protokoll um beispielsweise das WWW in den Webbrowser zu laden}
}

\newglossaryentry{FTP}{%
    name={FTP},
    description={File Transfer Protocol. Protokoll zur Übermittlung von Daten}
}

\newglossaryentry{API}{%
    name={API},
    description={Application-Programming-Interface. Programmschnittstelle, die Daten in maschinenlesbarer Form darstellt}
}

\newglossaryentry{HTML}{%
    name={HTML},
    description={Hypertext Markup Language. Beschreibungssprache die häufig für die Inhaltsbeschreibung von Webseiten verwendet wird}
}

\newglossaryentry{TESAC}{%
  name={TESAC},
  description={Wissenschaftliches Datenformat zur Übertragung von Sensordaten}
}

\newglossaryentry{BUFR}{%
  name={BUFR},
  description={Binäres Datenformat zur Übetragung meterologischer Daten}
}


\newglossaryentry{Framework}{%
    name={Framework},
    description={Erweiterbare Softwarebibliothek zur Erstellung von Systemen. Funktioniert nach dem Hollywood-Prinzip (don't call us, we call you)}
}

\newglossaryentry{Julian Day}{%
    name={Julianischen Datum},
    description={(julian day) In der Wissenschaft gebräuchliches Datumsformat über die Tageszählung ab einem fixen Startdatum (meist 1. Januar 4713 v. Chr.)}
}
