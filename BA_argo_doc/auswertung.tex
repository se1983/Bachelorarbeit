\section{Demonstration und Auswertung}


\subsection{Geschaffene Lösung}

Im Rahmen dieser Arbeit wurde eine Webplattform geschaffen, die es erlaubt, die Daten des Argo-Programms durch eine Kartendarstellung der Weltmeere zu identifizieren und die Daten zur Anzeige zu bringen.
Dafür wurden die Daten aus den Datensätzen extrahiert und in ein Relationales Schema überführt. Die Webpräsenz nutzt die Datenbank um die Inhalte darzustellen. In Abbildung \ref{fig:argodataWeb} ist ein Screenshot der hier erarbeiteten Lösung zu sehen.


...

\begin{figure}[H]
 \centering
 \includegraphics[width=\textwidth]{pix/argodata_complete.png}
 % argodata_complete.png: 1872x1026 px, 96dpi, 49.52x27.14 cm, bb=0 0 1404 769
 \caption{Die Webpräsenz von Argo-Data}
 \label{fig:argodataWeb}
\end{figure}

\subsubsection{Erweiterungsmöglichkeiten}




\subsubsection{Kritik}






Der echte Zug des Wissens ist nichts Statisches, das man anhalten und in Teile zerlegen kann. Er ist immer in Fahrt. Auf einem Gleis namens Qualität. Und die Lok und die 120 Güterwagen fahren nie woanders hin, als wo das Gleis der Qualität sie hinführt.
\\
\hrulefill \vspace{0.3cm}
\textbf{Robert M. Pirsing} -- \textit{Zen und die Kunst ein Motorrad zu warten}

