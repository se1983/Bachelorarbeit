

\begin{titlepage}
    \begin{addmargin}[4cm]{-1cm}
        \centering
        \hfill\includegraphics[width=4cm]{pix/S07_HTW_Berlin_Logo_pos_GRAU_RGB.jpg}\par
        \vspace{4\baselineskip}
        {\Huge 
        \rmfamily{Konzeption und prototypische Entwicklung einer Webanwendung  zur Aggregation und Analyse wissenschaftlicher Daten des Argo-Projektes}\par}
%        {\Large Zeitraum 01.01.2001 - 02.02.2002 }
        \vspace{4\baselineskip}
    
        \par
        {\Large Sebastian Schmid \\ S0543196 \par}
        \vfill
        {Prof. Dr. Christin Schmidt \\ Prof. Dr.-Ing. Hendrik Gärtner} 
        \vfill
        Bachelorarbeit zur Erlangung des akademischen Grades\\ Bachelor of Science (B.Sc.)\par
        {Fachbereich Wirtschaftswissenschaften II \\ Studiengang Angewandte Informatik \\ an der Hochschule für Technik und Wirtschaft Berlin}
    \end{addmargin}
\end{titlepage}

\newpage

\section*{Erklärung der Urheberschaft}\thispagestyle{empty}

Ich erkläre hiermit an Eides statt, dass ich die vorliegende Arbeit
ohne Hilfe Dritter und ohne Benutzung anderer als der angegebenen
Hilfsmittel angefertigt habe; die aus fremden Quellen direkt oder
indirekt übernommenen Gedanken sind als solche kenntlich gemacht. Die
Arbeit wurde bisher in gleicher oder ähnlicher Form in keiner anderen
Prüfungsbehörde vorgelegt und auch noch nicht veröffentlicht.


\vspace{4cm}

\hspace{2cm} Ort, Datum \hfill Unterschrift \hspace{2cm}


\newpage

\section*{Danksagung}

An dieser Stelle möchte ich  meiner Betreuerin Frau Prof. Dr. Christin Schmidt danken. Diese unterstützte mich stets tatkräftig. Auch Herrn Prof. Dr.-Ing. Gärtner möchte ich dafür danken, dass er sich dazu bereit erklärt hat, diese Arbeit zu betreuen. 

Allen Teilnehmern an der Usability-Umfrage möchte ich ebenso meinen Dank aussprechen.

Außerdem möchte ich auch  meinen Eltern danken. Auch wenn diese heute nicht mehr unter uns sind, so waren sie es doch, die den Keim gelegt haben, der mich dazu befähigte, den Weg bis zu dieser Stelle zu gehen.

Meine Freundin begleitete mich auf dem Weg dieser Ausarbeitung und stand mir stets mit Rat und Tat zur Seite. Da diese in ganz besonderer Weise an dieser ereignisreichen Zeit teilhaben durfte, ist ihr ganz besonders zu danken.

Zu danken habe ich auch Robert M. Pirsig. Sein Buch "`Zen und die Kunst ein Motorrad zu warten"' begleitete mich bei der Entwicklung dieser Arbeit und inspirierte mich immer wieder aufs Neue. Ich könnte mir kein besseres Buch als Begleitung für eine schriftliche Ausarbeitung vorstellen.



